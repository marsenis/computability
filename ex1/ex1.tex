% --- To be compiled with XeLaTeX ---
% ---      Encoding: UTF-8        ---

\documentclass[a4paper, oneside, 11pt]{article}

%fontspec package provides a configurable interface for font selection, and allows complex font choices to be named and later reused. It's needed for XeLaTeX
\usepackage[cm-default]{fontspec}

% Unicode support
\usepackage{xunicode}
\usepackage{xltxtra}

% Default words and phrases in Greek (e.g. 'Περίληψη' instead of 'Abstract'). Also contains hyphenation rules for Greek Language
\usepackage{xgreek}

% Mathematical fonts, theorems etc.
\usepackage{amsfonts}
\usepackage{amsmath}
\usepackage{amsthm}

% Default page layout for consuming a larger portion of the page.
\usepackage{fullpage}

% Greek fonts (Computer Modern)
\setmainfont[Mapping=tex-text]{CMU Serif}

% Auxiliary commands
\newcommand{\red}{\leq_{\text{m}}}

\newtheorem{thm}{Θεώρημα}
\newtheorem{lm}[thm]{Λήμμα}

\theoremstyle{definition}
\newtheorem{defn}[thm]{Ορισμός}

\begin{document}

\section*{Άσκηση 1}

\begin{itemize}
   \item \textbf{Ψευδής}
   
         Έστω $x$ τέτοιο ώστε $M_x$ να είναι μια TM η οποία δεν τερματίζει για καμία
         είσοδο.  Τότε έχουμε $\text{Dom}(\phi_x) = \emptyset$ το οποίο είναι
         αναδρομικό σύνολο (το αποφασίζει η μηχανή Turing που απορρίπτει κάθε
         είσοδο), δηλαδή $x \in R$. Όμως από τον ορισμό της $M_x$ έχουμε $M_x(x)
         \uparrow$, δηλαδή $x \notin K$.

   \item \textbf{Ψευδής}

         Έστω $x$ τέτοιο ώστε $M_x$ να είναι μια TM η οποία τερματίζει και
         αποδέχεται κάθε είσοδο.
         Τότε έχουμε $\text{Dom}(\phi_x) = \mathbb{N}$ το οποίο είναι
         αναδρομικό σύνολο (το αποφασίζει η μηχανή Turing που αποδέχεται κάθε
         είσοδο), δηλαδή $x \in R$. Όμως από τον ορισμό της $M_x$ έχουμε $M_x(x)
         \downarrow$, δηλαδή $x \in K$ συνεπώς $x \in R \cap K$.

   \item \textbf{Ψευδής}

         Στα παρακάτω θα χρησιμοποιήσουμε το εξής πρόβλημα:

         \[ \text{HALT} = \{ \langle M, w \rangle\ |\ M(w) \downarrow \} \]

         για το οποίο γνωρίζουμε ότι $\text{HALT } \in \text{ ER}$ και
         $\text{HALT } \notin \text{REC}$ επομένως
         $\overline{\text{HALT} } \notin \text{ ER}$ και θα κάνουμε την
         αναγωγή $\overline{\text{HALT} } \red R$ από την οποία
         προκύπτει ότι $R \notin \text{ER}$. Συνεπώς και $R \cup K \notin \text{ER}$.

         \begin{lm}
            $\overline{\text{HALT}} \red R$.
         \end{lm}
         \begin{proof}
            Θα πρέπει να βρούμε μια συνάρτηση $f : \mathbb{N} \rightarrow \mathbb{N}$
            η οποία να είναι υπολογίσιμη και για την οποία
            να ισχύει:
            $\langle M, w \rangle \in \overline{\text{HALT}} \Leftrightarrow
               f(\langle M, w \rangle) \in R$.

            Αυτή η $f$ θα είναι η συνάρτηση που υπολογίζει η παρακάτω μηχανή Turing:

            \begin{itemize}
               \item $F = $`` Για είσοδο $\langle M, w \rangle$:
               \begin{enumerate}
               \item Δημιούργησε την περιγραφή και βρες
                     τον αριθμό G\"{o}del $g$ της μηχανής Turing $T$:\\
                     $T = $ `Για είσοδο $\langle T_1, x \rangle$:
                     \begin{enumerate}
                     \item Προσομοίωσε παράλληλα την $M$ με είσοδο $w$ και
                           την $T_1$ με είσοδο $x$.
                     \item Αν τερματίσουν και οι δύο, τότε επέστρεψε το $T_1(x)$.'
                     \end{enumerate}
               \item Γράψε $g$ στην ταινία εξόδου.''
               \end{enumerate}
            \end{itemize}

            Έστω ότι $\langle M, w \rangle \in \overline{\text{HALT}}$, δηλαδή $M(w)
            \uparrow$, τότε η μηχανή $T$ δεν τερματίζει για καμία είσοδο άρα
            $f(\langle M, w \rangle) = g$ όπου $g$ τέτοιο ώστε $\text{Dom}(\phi_g) =
            \emptyset \in \text{REC}$, συνεπώς $g \in R$.

            Αντίστροφα, αν $\langle M, w \rangle \in \text{HALT}$ τότε η συνάρτηση $f$
            επιστρέφει ένα $g$ για το οποίο $\text{Dom}(\phi_g) = \text{HALT} \notin
            \text{REC}$, άρα $g \in \overline{R}$.
         \end{proof}
         
\end{itemize}

\end{document}
