% --- To be compiled with XeLaTeX ---
% ---      Encoding: UTF-8        ---

\documentclass[a4paper, oneside, 11pt]{article}

%fontspec package provides a configurable interface for font selection, and allows complex font choices to be named and later reused. It's needed for XeLaTeX
\usepackage[cm-default]{fontspec}

% Unicode support
\usepackage{xunicode}
\usepackage{xltxtra}

% Default words and phrases in Greek (e.g. 'Περίληψη' instead of 'Abstract'). Also contains hyphenation rules for Greek Language
\usepackage{xgreek}

% Mathematical fonts, theorems etc.
\usepackage{amsfonts}
\usepackage{amsmath}
\usepackage{amsthm}

% Default page layout for consuming a larger portion of the page.
\usepackage{fullpage}

% Greek fonts (Computer Modern)
\setmainfont[Mapping=tex-text]{CMU Serif}

% Auxiliary commands
\newcommand{\red}{\leq_{\text{m}}}

\newtheorem{thm}{Θεώρημα}
\newtheorem{lm}[thm]{Λήμμα}

\theoremstyle{definition}
\newtheorem{defn}[thm]{Ορισμός}

\begin{document}

\section*{Άσκηση 1}

\begin{itemize}
   \item \textbf{Ψευδής}
   
         Έστω $x$ τέτοιο ώστε $M_x$ να είναι μια TM η οποία δεν τερματίζει για καμία
         είσοδο.  Τότε έχουμε $\text{Dom}(\phi_x) = \emptyset$ το οποίο είναι
         αναδρομικό σύνολο (το αποφασίζει η μηχανή Turing που απορρίπτει κάθε
         είσοδο), δηλαδή $x \in R$. Όμως από τον ορισμό της $M_x$ έχουμε $M_x(x)
         \uparrow$, δηλαδή $x \notin K$.

   \item \textbf{Ψευδής}

         Έστω $x$ τέτοιο ώστε $M_x$ να είναι μια TM η οποία τερματίζει και
         αποδέχεται κάθε είσοδο.
         Τότε έχουμε $\text{Dom}(\phi_x) = \mathbb{N}$ το οποίο είναι
         αναδρομικό σύνολο (το αποφασίζει η μηχανή Turing που αποδέχεται κάθε
         είσοδο), δηλαδή $x \in R$. Όμως από τον ορισμό της $M_x$ έχουμε $M_x(x)
         \downarrow$, δηλαδή $x \in K$ συνεπώς $x \in R \cap K$.

   \item \textbf{Ψευδής}

         Στα παρακάτω θα χρησιμοποιήσουμε το εξής πρόβλημα:

         \[ \text{HALT} = \{ \langle M, w \rangle\ |\ M(w) \downarrow \} \]

         για το οποίο γνωρίζουμε ότι $\text{HALT } \in \text{ ER}$ και
         $\text{HALT } \notin \text{REC}$ επομένως
         $\overline{\text{HALT} } \notin \text{ ER}$ και θα κάνουμε την
         αναγωγή $\overline{\text{HALT} } \red R$ από την οποία
         προκύπτει ότι $R \notin \text{ER}$. Συνεπώς και $R \cup K \notin \text{ER}$.

         \begin{lm}
            $\overline{\text{HALT}} \red R$.
         \end{lm}
         \begin{proof}
            Θα πρέπει να βρούμε μια συνάρτηση $f : \mathbb{N} \rightarrow \mathbb{N}$
            η οποία να είναι υπολογίσιμη και για την οποία
            να ισχύει:
            $\langle M, w \rangle \in \overline{\text{HALT}} \Leftrightarrow
               f(\langle M, w \rangle) \in R$.

            Αυτή η $f$ θα είναι η συνάρτηση που υπολογίζει η παρακάτω μηχανή Turing:

            \begin{itemize}
               \item $F = $`` Για είσοδο $\langle M, w \rangle$:
               \begin{enumerate}
               \item Δημιούργησε την περιγραφή και βρες
                     τον αριθμό G\"{o}del $g$ της μηχανής Turing $T$:\\
                     $T = $ `Για είσοδο $\langle T_1, x \rangle$:
                     \begin{enumerate}
                     \item Προσομοίωσε παράλληλα την $M$ με είσοδο $w$ και
                           την $T_1$ με είσοδο $x$.
                     \item Αν τερματίσουν και οι δύο, τότε επέστρεψε το $T_1(x)$.'
                     \end{enumerate}
               \item Γράψε $g$ στην ταινία εξόδου.''
               \end{enumerate}
            \end{itemize}

            Έστω ότι $\langle M, w \rangle \in \overline{\text{HALT}}$, δηλαδή $M(w)
            \uparrow$, τότε η μηχανή $T$ δεν τερματίζει για καμία είσοδο άρα
            $f(\langle M, w \rangle) = g$ όπου $g$ τέτοιο ώστε $\text{Dom}(\phi_g) =
            \emptyset \in \text{REC}$, συνεπώς $g \in R$.

            Αντίστροφα, αν $\langle M, w \rangle \in \text{HALT}$ τότε η συνάρτηση $f$
            επιστρέφει ένα $g$ για το οποίο $\text{Dom}(\phi_g) = \text{HALT} \notin
            \text{REC}$, άρα $g \in \overline{R}$.
         \end{proof}
         
\end{itemize}

\section*{Άσκηση 2}

\begin{lm}
Το πρόβλημα ΔΙΑΨΕΥΣΗ GOLDBACH είναι αναγνωρίσιμο.
\end{lm}
\begin{proof}
Θεωρούμε τη μηχανή Turing με την εξής λειτουργία: Αν $n$ είναι η είσοδος της μηχανής,
δοκιμάζει με τη σειρά τους ακεραίους $k\geq n$. Για κάθε τέτοιο αριθμό, δοκιμάζει όλα
τα ζευγάρια ακεραίων $p,q\leq 2k$, τα οποία προφανώς είναι πεπερασμένα για σταθερό $k$,
και ελέγχει αν τα $p$ και $q$ είναι πρώτοι (δοκιμάζοντας όλους τους πιθανούς διαιρέτες)
και αν $2k = p + q$. Αν για κάποιο $k$ δεν βρεθούν $p$ και $q$ που να ικανοποιούν τις
παραπάνω ιδιότητες, τότε η μηχανή Turing απαντάει ΝΑΙ στο πρόβλημα ΔΙΑΨΕΥΣΗ GOLDBACH.
Πραγματικά, αν υπάρχει ακέραιος $k\geq n$ που να διαψεύδει την εικασία του Goldbach,
τότε δεν θα υπάρχει ζευγάρι πρώτων $p,q\leq 2k$ έτσι ώστε $2k = p+q$. Η μηχανή Turing
θα φτάσει σε αυτό το $k$ σε πεπερασμένο χρόνο, άρα σε πεπερασμένο χρόνο θα απαντήσει
ΝΑΙ. Συνεπώς το πρόβλημα ΔΙΑΨΕΥΣΗ GOLDBACH είναι αναγνωρίσιμο.
\end{proof}

\begin{lm}
Αν το πρόβλημα ΕΠΑΛΗΘΕΥΣΗ GOLDBACH είναι αναγνωρίσιμο, τότε θα είναι και διαγνώσιμο.
\end{lm}
\begin{proof}
Θεωρούμε τη μηχανή Turing $M_1$, η οποία αναγνωρίζει το πρόβλημα ΕΠΑΛΗΘΕΥΣΗ GOLDBACH,
και τη μηχανη $M_2$, η οποία αναγνωρίζει το πρόβλημα ΔΙΑΨΕΥΣΗ GOLDBACH. Τώρα, έστω
μηχανή $M$ η οποία εξομοιώνει τις μηχανές $M_1$ και $M_2$ για ένα βήμα την
καθεμιά εναλλάξ. Αν κάποια στιγμή αποδεχθεί η $M_1$, τότε η $M$ τερματίζει και δίνει
απάντηση ΝΑΙ για το πρόβλημα ΕΠΑΛΗΘΕΥΣΗ GOLDBACH. Αν, από την άλλη, κάποια στιγμή
αποδεχθεί η $M_2$, τότε η $M$ τερματίζει και δίνει απάντηση ΟΧΙ για το πρόβλημα 
ΕΠΑΛΗΘΕΥΣΗ GOLDBACH. Επειδή κάποιο από τα δύο παραπάνω σενάρια θα συμβεί σε πεπερασμένο
χρόνο, η μηχανή $M$ τερματίζει σε πεπερασμένο χρόνο και αποφασίζει το πρόβλημα
ΕΠΑΛΗΘΕΥΣΗ GOLDBACH. Αυτό σημαίνει ότι το πρόβλημα αυτό είναι διαγνώσιμο.
\end{proof}

\section*{Άσκηση 3}

Όπως έχουμε δεί και στο μάθημα, αν μια μηχανή Turing είναι περιορισμένη στις πρώτες
$n$ θέσεις τότε υπάρχουν $t(n) = |Q| \cdot n \cdot |\Sigma|^n$ πιθανές καταστάσεις στις
οποίες μπορεί να βρίσκεται κάθε στιγμή. Αν βρεθεί σε κάποια από αυτές τις καταστάσεις
για δεύτερη φορά τότε θα πέσει σε άπειρο βρόχο.

Επομένως για μια μή περιορισμένη TM, ο μόνος τρόπος να φτάσει στη θέση $n+1$ είναι να
το κάνει σε λιγότερο από $t$ βήματα.

Φτιάχνουμε λοιπόν μια μηχανή Turing T η οποία για είσοδο $\langle M, w, n_1, n_2
\rangle$ προσομοιώνει την $M$ με είσοδο $w$ για $t(\min(n_1, n_2)-1)$ βήματα και αν η
$M$ φτάσει στη θέση $\min(n_1, n_2)$ αποδέχεται διαφορετικά απορρίπτει.

Η T αποφασίζει το πρόβλημα ΠΕΠΕΡΑΣΜΕΝΗ ΜΝΗΝΗ αφού πάντα τερματίζει και αποδέχεται ανν
η $M$ επισκεφτεί την θέση $\min(n_1, n_2)$ το οποίο είναι ισοδύναμο με το να
επισκεφτεί κάποια εκ των θέσεων $n_1, n_2$ αφού για να πάει μια μηχανή Turing από μία
θέση σε μία άλλη θα πρέπει να περάσει από όλες τις ενδιάμεσες.

\section*{Άσκηση 4}

Έστω οποιαδήποτε μηχανή Turing $M_1$ και είσοδος $x$. Θα δείξουμε ότι αν η γλώσσα $L$ ήταν αναδρομική,
τότε θα μπορούσαμε να ξέρουμε σε πεπερασμένο χρόνο αν η $M_1$ με είσοδο $x$ τερματίζει ή όχι.
Έστω $M$ μηχανή Turing που αποφασίζει τη γλώσσα $L$. Θα κατασκευάσουμε μία μηχανή $M_2$, η οποία θα είναι
ίδια με την $M_1$, με τη διαφορά ότι για κάθε μετάβαση από μία κατάσταση $q$ σε μία τελική κατάσταση $q'$ 
με τον κανόνα $(q',\gamma_1',m)=\delta_1(q,\gamma)$, για κάποιο σύμβολο $\gamma$, θα αλλάξουμε τη συνάρτηση
μετάβασης σε $(q',\gamma_2',m)=\delta_2(q,\gamma)$, όπου $\gamma_1'\neq \gamma_2'$. Δηλαδή απλώς θα αλλάξουμε το σύμβολο που γράφεται
στην ταινία πριν από μια τελική κατάσταση.
\begin{lm}
Η $M_1$ δεν τερματίζει με είσοδο $x$ αν και μόνο αν οι $M_1$ και $M_2$ κατά τη λειτουργία τους με είσοδο
$x$ γράφουν μετά από τον ίδιο αριθμό βημάτων το ίδιο σύμβολο στην ταινία τους.
\end{lm}
\begin{proof}
...
\end{proof}

\section*{Άσκηση 9}

Έστω $T_1, T_2$ οι μηχανές που αναγνωρίζουν τις γλώσσες $\overline{A}, \overline{B}$
αντίστοιχα.

Δημιουργούμε την μηχανή $T$:

\begin{itemize}
\item $T = $`` Για είσοδο $x$:
      \begin{enumerate}
         \item Τρέξε παράλληλα την $T_1$ με είσοδο $x$ και την $T_2$ με είσοδο $x$
               μέχρι κάποια εκ των δύο να τερματίσει.
         \item Αν τερμάτισε η $T_1$ τότε επέστρεψε $\overline{T_1(x)}$.
         \item Αν τερμάτισε η $T_2$ τότε επέστρεψε $T_2(x)$.''
      \end{enumerate}
\end{itemize}

Θα δείξουμε ότι $C = L(T)$ είναι η γλώσσα που αναζητούμε.

Αρχικά παρατηρούμε ότι η $T$ πάντα τερματίζει αφού $A \cap B = \emptyset
\Leftrightarrow \overline{A} \cup \overline{B} = \Sigma^*$. Άρα για κάθε $x \in
\Sigma^*$, έχουμε είτε $x \in \overline{A}$ είτε $x \in \overline{B}$ συνεπώς μία εκ
των $T_1, T_2$ θα αποδεχτεί το $x$ κι έτσι $C \in \text{REC}$.

\begin{itemize}
\item $A \subseteq C$:
      
      Έστω $x \in A$. Αφού $A \cap B = \emptyset$ θα έχουμε $x \notin B
      \Rightarrow x \in \overline{B}$.

      Εκτελούμε τη μηχανή $T$ με είσοδο το $x$. Επειδή $x \in \overline{B}$ ξέρουμε
      ότι η $T_2$ θα τερματίσει κάποια πεπερασμένη χρονική στιγμή και θα αποδεχτεί.
      Αν τερματίσει πριν την $T_1$ τότε θα εκτελεστεί η γραμμή 3 και έτσι και η $T$
      αποδέχεται. Διαφορετικά, αν η $T_1$ τερματίσει πρώτη θα
      πρέπει αναγκαστικά να απαντήσει reject, συνεπώς λόγω της γραμμής 2 η $T$ θα
      αποδεχθεί.

      Σε κάθε περίπτωση η $T$ αποδέχεται το $x$, συνεπώς $x \in C$.

\item $B \subseteq \overline{C}$

      Έστω $x \in B$. Αντίστοιχα με προηγουμένως εκτελούμε την $T$ με είσοδο $x$ και
      επειδή $x \notin A$ η $T_1$ θα αποδεχτεί σε πεπερασμένο πλήθος βημάτων. Αν η
      $T_1$ τερματίσει πρώτη τότε η $T$ απορρίπτει ενώ αν η $T_2$ τερματίσει πρώτη
      αναγκαστικά θα απορρίψει και έτσι και η $T$ θα απορρίψει.

      Σε κάθε περίπτωση $x \notin C$.
\end{itemize}

\section*{Άσκηση 11}

Έστω $L$ μια άπειρη αναδρομικά απαριθμήσιμη γλώσσα. Συνεπώς υπάρχει ένας απαριθμητής
$T$ ο οποίος τυπώνει στην έξοδό του όλες τις λέξεις της γλώσσας με αυθαίρετη σειρά
και πιθανόν με επαναλήψεις.

Αρχικά παρατηρούμε ότι μπορούμε να δημιουργήσουμε μια μηχανή $T'$ παρόμοια με την $T$
με τη διαφορά ότι θα διατηρεί μια λίστα με τις λέξεις που έχει τυπώσει και πριν
τυπώσει μια λέξη να ελέγχει ότι όντως είναι η πρώτη φορά που την τυπώνει. H $T'$
συνεχίζει να είναι απαριθμητής της $L$ αφού αν η $T$ τύπωνε μία λέξη $w$ σε
πεπερασμένο αριθμό βημάτων το ίδιο θα κάνει και η $T'$.

Άρα θεωρούμε χωρίς βλάβη της γενικότητας ότι η $T$ τυπώνει κάθε λέξη της $L$ ακριβώς
μία φορά οπότε δημιουργούμε τις μηχανές $T_1, T_2$ οι οποίες κάνουν ό,τι κάνει και η
$T$ με τη διαφορά ότι τυπώνουν κάθε δεύτερη λέξη ξεκινώντας από την πρώτη ή τη
δεύτερη αντίστοιχα.

Άρα οι $T_1, T_2$ απαριθμούν τις άπειρες γλώσσες $A, B$ για τις οποίες $A \cap B =
\emptyset$ και $A \cup B = L$.

\section*{Άσκηση 13}

Έστω η ΤΜ:

\begin{itemize}
\item $T = $``Για είσοδο $n$:
      \begin{enumerate}
      \item Για $1 \leq i \leq n$:
            \begin{enumerate}
            \item Προσομοίωσε την $M_i$ με είσοδο $n$ και τοποθέτησε το αποτέλεσμα σε
                  μία ειδική ταινία.
            \end{enumerate}
      \item Υπολόγισε και επέστρεψε το άθροισμα όλων των τιμών της ειδικής ταινίας.''
      \end{enumerate}
\end{itemize}

Θα δείξουμε ότι η $T$ υπολογίζει την $f$. Πράγματι, αν για κάποιο $n$ η $f$ ορίζεται,
δηλαδή ισχύει
$n \in \text{Dom}(\phi_i)$ για κάθε $1 \leq i \leq n$ τότε η 
κάθε προσομοίωση στο βήμα 1 τερματίζει μετά από πεπερασμένο αριθμό βημάτων και έτσι η
$T$ τερματίζει και επιστρέφει $T(n) = f(n)$

\end{document}
