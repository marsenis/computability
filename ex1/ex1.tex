% --- To be compiled with XeLaTeX ---
% ---      Encoding: UTF-8        ---

\documentclass[a4paper, oneside, 11pt]{article}

%fontspec package provides a configurable interface for font selection, and allows complex font choices to be named and later reused. It's needed for XeLaTeX
\usepackage[cm-default]{fontspec}

% Unicode support
\usepackage{xunicode}
\usepackage{xltxtra}

% Default words and phrases in Greek (e.g. 'Περίληψη' instead of 'Abstract'). Also contains hyphenation rules for Greek Language
\usepackage{xgreek}

% Mathematical fonts, theorems etc.
\usepackage{amsfonts}
\usepackage{amsmath}
\usepackage{amsthm}

% Default page layout for consuming a larger portion of the page.
\usepackage{fullpage}

% Greek fonts (Computer Modern)
\setmainfont[Mapping=tex-text]{CMU Serif}

% Auxiliary commands
\newcommand{\HRule}{\rule{\linewidth}{0.5mm}}

\begin{document}

\section*{Άσκηση 1}

\begin{itemize}
   \item \textbf{Ψευδής}
   
         Έστω $x$ τέτοιο ώστε $M_x$ να είναι μια TM η οποία δεν τερματίζει για καμία
         είσοδο.  Τότε έχουμε $\text{Dom}(\phi_x) = \emptyset$ το οποίο είναι
         αναδρομικό σύνολο (το αποφασίζει η μηχανή Turing που απορρίπτει κάθε
         είσοδο), δηλαδή $x \in R$. Όμως από τον ορισμό της $M_x$ έχουμε $M_x(x)
         \uparrow$, δηλαδή $x \notin K$.

   \item \textbf{Ψευδής}

         Έστω $x$ τέτοιο ώστε $M_x$ να είναι μια TM η οποία τερματίζει και
         αποδέχεται κάθε είσοδο.
         Τότε έχουμε $\text{Dom}(\phi_x) = \mathbb{N}$ το οποίο είναι
         αναδρομικό σύνολο (το αποφασίζει η μηχανή Turing που αποδέχεται κάθε
         είσοδο), δηλαδή $x \in R$. Όμως από τον ορισμό της $M_x$ έχουμε $M_x(x)
         \downarrow$, δηλαδή $x \in K$ συνεπώς $x \in R \cap K$.

   \item \textbf{Ψευδής}

         Στα παρακάτω θα χρησιμοποιήσουμε το εξής πρόβλημα:

         \[ \text{HALT} = \{ \langle M \rangle\ |\ \exists x: M(x) \downarrow \} \]

         για το οποίο γνωρίζουμε ότι $\text{HALT } \in \text{ ER}$ και
         $\overline{\text{HALT} } \notin \text{ ER}$.

         ...
         
\end{itemize}

\end{document}
